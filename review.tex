\chapter{Literature Review}

\section{Respiration in sport}
Minute ventilation is deftined as the amount of air inhaled into body in one minute. It is expressed as the product of tidal volume and respiratory frequency. Tidal volume refers to the amount of air inhaled per breath. Respiration frequency is the number of breaths taken in one minute \cite{Zuurbier2009MinuteStudy}. The research conducted by Nicolo et al showed that breathing pattern is affected by the setting of training, such as intensity, duration and resting time \cite{Nicolo2018RespiratoryInterdependence}. As exercise begins, the increase in ventilation is directly proportional to the intensity of exercise in order to meet metabolic needs. The respiratory frequency may increase from a typical rest rate of 15 breaths per minute to 40-50 breaths per minute. Unlike other physiological measures, such as heart rate, blood pressure, and ECG signals, respiratory frequency is usually neglected during sports training monitoring. Respiratory frequency and physical effort has the strongest correlation compared to other physiological measures according to Nicolo et al \cite{Nicolo2017RespiratoryMeasure}. Their study also showed that by extracting information from the respiration frequency, it is possible to analyse athletes' performance from an more advanced perspective.

\section{Respiration in clinical monitoring}
Breathing is an essential activity for our body which allows the air exchange between the human body and the external environment. There are mainly two functions supported by the respiratory system, providing oxygen to the tissues of body and eliminate carbon dioxide from the tissues of the body. 

Two common forms for breathing includes chest breathing and abdominal breathing. One complete breath cycle includes an inhalation and an exhalation. Respiratory frequency refers to the number of breaths per minute. The respiratory frequency of adults at rest is approximately 12-18 breaths/minute and 30-40 breaths/minute for children at rest. Adults who have a respiratory frequency of more than 20 breaths/minute may be in unhealthy conditions. A respiratory frequency of over 24 breaths/minute indicates serious health issues. It was also reported that a respiratory frequency over 27 breaths/minute is very likely to associate with cardiac arrest \cite{Cretikos2008OfMBA}.  Therefore, by detecting the respiratory frequency of the athletes after training, it is beneficial to ensure their health and timely detection of abnormal physical conditions. Therefore, respiratory frequency is an informative indicator in clinical monitoring.

\section{Methods of measuring respiratory frequency}
Most of the commercially available solutions are invasive and importable causing unpromising uses during outdoor exercise. Due to the approach applied in the solutions, different solutions are suitable for certain scenarios. There are two main categories approaches widely used to measure respiratory frequency, direct measuring and indirect measuring. The direct measuring considers measuring the parameters which are directly associated with breathing, such as breathing airflow, breathing sound, air temperature, air humidity and air component. On the other hand, the parameters which are taken into account by indirect measuring includes chest wall movement, cardiac activity (bio-signals like electrocardiography and photoplethysmograph) \cite{Massaroni2019Contact-BasedRate}. It is necessary to consider the use cases while choose appropriate solutions.

The tradeoff between accuracy and usability of the different approaches were compared in the study conducted by Massaroni et al \cite{Massaroni2019Contact-BasedRate}. Breathing airflow based solutions provide the best performance in terms of accuracy among all solutions, but they are invasive and inconvenient to use in unstructured environments. Moreover, solutions based on air temperature, air humility and air component  is capable of providing sufficient accuracy, but are greatly affected by the environment, whether indoors or outdoors. In contrast, it is recommended to use the solutions based on chest motion and cardiac activity for outdoor uses but the performance might be affected by the motion of body. Respiratory sound-based solutions are of interest to this study, providing adequate measurement performance both indoors and outdoors, but are susceptible to environmental disturbances. 

There are some implementations applying the idea of indirect measuring. For example, the system proposed by Mason suggested an non-invasive approach that taking advantage of the fusion of multiple physiological measures to infer the waveform of respiration to achieve detecting respiratory frequency \cite{Mason2002SignalMonitoring}. However, additional sensors and equipment are required to acquire the desired parameters. As the popularity of smartphones grows, more and more sport and medical applications appear on mobile platforms. Breathing sound is the easiest to collect indoors and outdoors among all of the solutions mentioned above because only a microphone is needed. The greatest concern of using breathing sound based solutions is the unexpected sounds in the environment which can dramatically affect the performance of the solution. Therefore, the performance of the breathing sound based solutions will improve if it is managed to reduce or even eliminate the impact caused from surrounding environment.

\section{Applications utilising respiratory sound}
A considerable amount of studies have proven that respiratory sound signal carries informative characteristics of human health that can be used to identify unhealthy subjects and even determine the severity of their disease. It has also shown that commercial headset's microphones or smartphones built-in microphones are enough to to capture respiratory-related features. The researches that utilise the acoustic respiration signal in various applications are reviewed as followed.

\subsection{Respiration pattern monitoring}
Fine breathing-related information can help to better understand the patient's breathing patterns. Recently there have been many studies focusing on providing prediction on respiration related events based on respiratory sound. As an alternative to the airflow based solutions in respiratory phase detection, the approach proposed by Yahya and Faezipour showed the capability of distinguishing inspiration phase and expiration phase using the acoustic respiration sound signal collected from the microphone placed in front of subject’s nose \cite{Yahya2014AutomaticCycles}. Also, Hamke et al developed an unsupervised machine learning model to process respiratory sound and to predict respiratory phase and respiratory depth by combining Linear Predictive Coefficients and Restricted Boltzmann Machine \cite{Hamke2019DetectingMachines}.

Finally, the study conducted by Nam et al showed the capability of detecting the respiratory frequency indoors using the acoustic respiration signals collected by the headset’s microphone and achieves high accuracy in estimating respiratory frequency at both low and high respiratory frequencies \cite{Nam2016EstimationHeadset}. However, the subjects were not breathing naturally during the experiment, but breathed at a fixed frequency under the instruction from the researchers. The breathing sound signal collected were very likely to be exaggerated because of the unnatural breathing behaviours. Nonetheless, the results still showed that acoustic respiration signal includes sufficient features that can be used to estimate the respiratory frequency. 

\subsection{Clinical monitoring}
Application using respiratory acoustic signals has diverse uses in the field of clinical monitoring. The system proposed by Niu et al was able to detect the presence of sputum in a patient’s trachea especially for the intubated subjects by analysing their respiratory sound. It is able to reduce the cost of medical staff since they must check the respiratory state periodically to prevent sputum deposition which could block the airways of patients and cause severe consequence \cite{Niu2019AState}. Besides, the system StressSense developed by Hong detects human stress level both in indoor and outdoor environments based on the human voice collected from the microphone embedded in mobile phones. It provides a relatively reliable and non-invasive monitoring solution for detecting stress in real-life situations \cite{Lu2012StressSense:Smartphones}.

The environment for monitoring the patient's sleep quality is fairly strict and the patient is required to sleep in the ward for one night, during which various physiological indicators are measured. Because of the different sleeping environments, this measurement may not differ from patient's actual sleep quality. It is, therefore, valuable to be able to monitor the quality of sleep in their home space. Using smartphone’s microphone to estimate respiratory frequency to determine patients’ sleep quality which provides an cheap and accessible alternative to polysomnography test \cite{Kaur2017UseRespiration}. Unlike most of the sleep quality monitoring applications only focus on detecting body movement, cough and snore, the system proposed by Ran et al aimed for providing fine-grained sleep monitoring by leveraging the respiratory sound collected from smartphone embedded microphone. The system not only detected snore, cough, turning over and getting up also provided respiratory frequency based on acoustic features extracted from the respiratory sound \cite{Ren2015Fine-grainedSmartphones}.

\subsection{Disease screening}
Stethoscope is one of the instruments widely used for initial diagnosis of respiratory system related disease. Stethoscope works by collecting the amplified sounds from the heart, lungs, arteries, veins, and other internal organs. In fact, using a microphone as a diagnostic instrument is to stimulate the method behind stethoscopes. It has been shown that respiratory sound collected from microphone has a great potential of being applied to the actual diagnosis for diseases in respiration system according to many recent studies. For example, by taking the advantage of the combination of perceptual and cepstral features, the respiratory sound based classifier is able to provide accurate diagnostic identification for breathing disorders, such as flu, pneumonia and bronchitis \cite{Lei2014Content-basedFeatures}. Moreover, the smartphone-based application developed by Uwaoma and Mansingh was able to recognise respiratory symptoms, such as cough, wheeze and stridor in real-time scenarios by using machine learning algorithms with respiratory signal \cite{Uwaoma2017OnAlgorithms}. 

In addition to being able to identify diseases associated with respiratory systems, the acoustic signals of breathing can also be used to distinguish the severity of certain diseases. According to Gogus et al, the approach that applying signal processing methods such as Wavelet Transforms and artificial neural networks on respiratory sound is able to classify subjects with no, mild, moderate and severe asthma \cite{Gogus2015ClassificationNetworks}. The system presented by Larson et al is an low-cost and reliable home-based alternative to clinical level spirometry which is able diagnose obstructive lung ailments in various degrees. It also uses the built-in microphone to estimate the respiratory flow rate during a spirometry test \cite{Boriello2012SpiroSmart}. Finally, by combining the acoustic signals of breathing with other types of signals, it even enables a wider range of diseases. The method proposed by Inoue et al proved to be capable of identifying patients with dysphagia by utilising the combination of respiratory flow, laryngeal motion and swallowing sounds \cite{Oku2018UsingSwallowing}.


