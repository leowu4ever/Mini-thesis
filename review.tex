\chapter{Review of Relevant Work}

\section{Respiration in sport monitoring}
As exercise begins, the increase in lung ventilation (breathing) is directly proportional to the intensity of exercise and metabolic needs. This is shown on the adjacent chart. Note that lung ventilation is expressed as the number of air inhalations (L / min) inhaled and exhaled per minute. Increase ventilation in two ways to meet your workout needs:1. The increase in "tidal volume" refers to the amount of air inhaled and exhaled per breath. This is similar to the "wheeze amount" in the cardiovascular system.2. The increase in "breathing or respiration rate" refers to the number of times each person completes inhalation and exhalation per minute. This is similar to the "heart rate" in the cardiovascular system. If exercise is intense, the breathing rate may increase from a typical rest rate of 15 breaths per minute to 40-50 breaths per minute. The most common measure of exercise and respiratory function is called VO2 (oxygen uptake). VO2 refers to the amount of oxygen absorbed and used by the body. With continuous exercises (duration longer than 1 minute), such as aerobic fitness, longer anaerobic fitness and lesser muscle endurance training, VO2 increases linearly with increasing exercise intensity. This is because as the movement continues, more and more oxygen is relied on to help provide energy. As the intensity of exercise continues to increase, one reaches the highest point beyond which oxygen consumption does not increase further. This point is called VO2 max, as shown in the figure below. The training type has medium intensity, lasts longer (longer than 1 minute), and has short or no rest throughout the process, which forms the so-called "EPOC". EPOC stands for "oxygen consumption after excessive exercise" and is related to the need to keep oxygen consumption at the end of the exercise at a rate greater than the resting rate to compensate for the oxygen "debt" generated at the beginning of the exercise. We will explain more about the chart below. It may take 10-20 minutes after exercise to restore normal breathing rate through hypertrophy training.

Unlike other physiological measures, such as heart rate, blood pressure, and ECG signals, respiratory frequency is usually neglected during sports training monitoring. The study conducted by Nicolo et al suggested that respiratory frequency and physical effort has the strongest correlation compared to other physiological measures.\cite{Nicolo2017RespiratoryMeasure} The study showed that by extracting information from the respiration frequency, it is possible to analyse athletes' performance from an more advanced perspective.


\section{Respiration in clinical monitoring}

Breathing is an essential activity for our body which allows the air exchange between the human body and the external environment. There are mainly two functions supported by the respiratory system, providing oxygen to the tissues of body and eliminate carbon dioxide from the tissues of the body. 

Two common forms for breathing includes chest breathing and abdominal breathing. One complete breath cycle includes an inhalation and an exhalation. The number of breaths per minute is called the respiratory rate. Respiratory frequency refers to the number of breaths per minute. The respiratory frequency of an adult at rest is approximately 12-18 breaths per minute and 30-40 breaths per minute for children. Adults who have a respiratory frequency of more than 20 breaths per minute may be in unhealthy conditions. A respiratory frequency of over 24 breaths per minute indicates serious health issues. It was also reported that a respiratory frequency over 27 breaths per minute is very likely to associate with cardiac arrest.\cite{Cretikos2008OfMBA} Therefore, by detecting the respiratory frequency of the athletes after training, it is beneficial to ensure their health and timely detection of abnormal physical conditions. Therefore, respiratory frequency is an indicator of some serious illnesses.

\section{Methods for respiratory frequency measurement}
Most of the solutions on the market are expensive and not suitable for everyday life, and some products are invasive, which is not available for outdoor use. The approaches widely used to measure respiratory frequency are mainly divided into two categories, one is direct measurement and the other is indirect measurement.The parameters considered in direct measurement includes respiratory airflow, respiratory sound, air temperature, air humidity and air components, chest-wall movement, cardiac activity (electrocardiography and photoplethysmography) are considered in indirect measurement. 
\cite{Massaroni2019Contact-BasedRate} 

These two types of solutions have very clear concerns. In the study completed by Massaroni, the accuracy and main advantages and disadvantages of these methods in different environments were compared. Breathing airflow-based solutions have the best performance in terms of accuracy, but they are invasive and inconvenient to use outdoors or in unstructured environments. Solutions based on air temperature, humidity and composition can provide sufficient accuracy, but are greatly affected by the environment, whether indoors or outdoors. However, solutions based on chest motion and cardiac activity can support acceptable measurements both indoors and outdoors, depending on the implementation, but if the accuracy is disturbed by motion. Respiratory sound-based solutions are also of interest to this research, providing adequate measurement performance both indoors and outdoors, but are susceptible to environmental disturbances. As the popularity of smartphones grows, more and more sports and medical-related applications appear on mobile platforms.

Of all the solutions mentioned above, only the breathing sound is the easiest to collect indoors and outdoors, because only the microphone is needed, and other parameters require additional sensors to acquire, which creates an additional barrier for measuring respiratory rate. . If it is possible to reduce noise from the surrounding environment, it is possible to expand the method based on breathing sound and improve accuracy. This is also one of the main goals of this research.


\section{Applications for extraction of respiration information}
A vast amount of studies have proven that the acoustic signals of breathing carry a lot of information about human health that can be used in classifiers to identify unhealthy people and determine the severity of the disease. In these studies, commercial headset microphones or smartphones built-in microphones. According to their experimental results, the signals collected by these devices are enough to be used to extract respiratory-related information. The researches that utilise the acoustic respiratory signal for different applications are reviewed as followed as guidelines for this study.

\subsection{Disease screening}
Stethoscopes are the most commonly used diagnostic tools for physicians to collect and amplify sounds from the heart, lungs, arteries, veins, and other internal organs. Most of this instrument is used only at the beginning of the diagnosis and does not represent the final diagnosis. The process by which the microphone collects the acoustic signal of the breath is the principle of mimicking the stethoscope.  

The respiratory sound based classifier is able to provide accurate diagnostic identification for breathing disorders, such as flu, pneumonia and bronchitis. By taking the advantage of the combination of enhanced perceptual and cepstral features, the classifier has a great potential in the diagnostic process.\cite{Lei2014Content-basedFeatures} The smartphone-based application developed by chinazunwa is able to recognise respiratory symptoms, such as cough, wheeze and stridor in real-time scenarios by using machine learning algorithms with respiratory signal.\cite{Uwaoma2017OnAlgorithms} The approach presented by Eric is an low-cost and reliable home-based alternative to clinical level spirometry which is able diagnose obstructive lung ailments in various degrees. It uses the built-in microphone to estimate the respiratory flow rate during a spirometry test.\cite{Boriello2012SpiroSmart} In addition to being able to identify diseases associated with different respiratory systems, the acoustic signals of breathing can also be used to distinguish the severity of certain diseases. In addition to being able to identify diseases related to different respiratory systems, according to Fatma, the approach that applying signal processing technique such as wavelet transforms with artificial neural networks on respiratory sound is able to classify subjects with no, mild, moderate and severe asthma.\cite{Gogus2015ClassificationNetworks} By combining the acoustic signals of breathing with other types of signals, it can identify a wider range of diseases. By utilising the combination of respiratory flow, laryngeal motion and swallowing sounds, The method proposed by Katsufumi Inoue claiming to achieve 86.0\% specificity and 82.4\% sensitivity on identifying patients with dysphagia.\cite{Oku2018UsingSwallowing}


\subsection{Clinical monitoring}
Application software using respiratory acoustic signals has a more diverse use scene in the field of Clinical monitoring. The system proposed by jinglong niu is able to detect the presence of sputum in a patient’s trachea especially for the intubated subjects by analysing their respiratory sound. It is able to reduce the cost of medical staff since they must check the respiratory state periodically to prevent sputum deposition which could block the airways of patients and cause severe consequence.\cite{Niu2019AState} It shows the value of continuous monitoring when the clinical monitoring is limited and expensive. 
Besides, the system StressSense developed by Hong detects human stress level both in indoor and outdoor environments based on the human voice collected from the microphone embedded in mobile phones. It provides a relatively reliable and non-invasive monitoring solution for detecting stress in real-life situations.\cite{Lu2012StressSense:Smartphones}

The environment for monitoring the patient's sleep quality is relatively strict, and the patient is required to sleep in the ward for one night, during which various physiological indicators are measured. Because of the different sleeping environments, this measurement may not represent the patient's actual sleep quality. It is, therefore, valuable to be able to monitor the quality of sleep in the home. Use smartphone’s microphone to estimate respiratory frequency to determine patients’ sleep quality which provides an cheap and accessible alternative to polysomnography test.\cite{Kaur2017UseRespiration} Unlike most of the sleep quality monitoring applications only focus on detecting body movement, cough and snore, the system proposed by yanzhi aims for providing fine-grained sleep monitoring by leveraging smartphones for collecting respiratory sound. The system not only detects snore, cough, turn over and get up also provide respiratory frequency based on acoustic features extracted from the respiratory sound.\cite{Ren2015Fine-grainedSmartphones}

Fine breathing-related information can help to better understand the patient's breathing patterns and methods. As an alternative to airflow based respiratory phase detection, the approach proposed by omar claims to achieve to 95\% accuracy in distinguishing inspiration and expiration phases using the acoustic respiratory sound signal collected from the microphone placed in front of subject’s nose.\cite{Yahya2014AutomaticCycles} Moreover, Eric developed an unsupervised machine learning approach to process respiratory sound to predict fine respiratory parameters respiratory depth or length by using linear predictive coefficients and restricted boltzmann machines. \cite{Hamke2019DetectingMachines}

A system claimed to be capable of detecting the respiratory frequency indoors using the acoustic respiration signals collected by the headset’s microphone and achieves high accuracy in estimating respiratory frequency at both low and high respiratory frequencies.\cite{Nam2016EstimationHeadset} However, the subjects did not breathe naturally in the experiment, but breathed at a fixed frequency under the instruction of the researchers. The breathing sound collected might be exaggerated because of the unnatural breathing approach. Moreover, most people do not breathe at a constant respiratory frequency. Nonetheless, this result still shows that the acoustic respiration signal includes sufficient features that can be used to estimate the respiratory frequency. Also, there are some methods of indirect measurements, such as a non-invasive approach that takes advantage of the fusion of multiple physiological measures to infer the waveform of respiration to achieve detecting respiratory frequency.\cite{Mason2002SignalMonitoring} However, it is still challenging to collect multiple physiological measures at the same time outdoors.

