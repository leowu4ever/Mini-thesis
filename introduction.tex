\chapter{Introduction}
\section{Aim}
The market has great demand for reliable, portable and affordable instruments that is capable for respiratory frequency measurement indoors and outdoors. This research aims deliver an alternative solution by developing a mobile application that utilises the acoustic respiration signals collected from a headset’s microphone to estimate the respiratory frequency and detect the respiratory phase in real-time. 

\section{Motivation}
In areas such as sports training monitoring and clinical monitoring, respiratory frequency is a significant measure because it contains informative indicators about athletes' performance and potential diseases. Efficient breathing approaches could contribute to athletes' performance, thus it is also essential to identify respiratory phases and analyse breathing patterns. Methods of estimating respiratory frequency widely applied include measuring the airflow of breathing, the expansion of chest wall and abdominal cavity, and the air temperature or humidity around nose \cite{Massaroni2019Contact-BasedRate}. Most of commercially available products are expensive and complicated, which are not suitable for daily-life usage. 

Being able to classify inhalation sound and exhalation sound is the foundation of estimating respiratory frequency and detecting respiratory phase. There have been many studies demonstrating that acoustic respiration signal contains enough features for being used in discriminating inhalation sound and exhalation sound \cite{Hamke2019DetectingMachines}\cite{Kaur2017UseRespiration}\cite{Ren2015Fine-grainedSmartphones}. It suggests a user-friendly, low-cost, and reliable solution to extract respiration-related information from respiration sound. 

\section{Objectives}
To investigate the feasibility of bringing a low-cost, intuitive and portable respiration monitoring solution. The major objective of this study is
\begin{itemize}
\item to design and develop a mobile-friendly solution that estimates respiratory frequency and determine respiratory phase and respiratory depth by utilising the acoustic respiration signal collected from headset’s microphone during running or resting.
\end{itemize}

It is also required to have respiratory frequency measures from other industry standard or gold standard instruments for labelling and referencing. The instruments used as comparison should not obstruct the respiratory sound collecting process, for example covering nose and month area. There are some solutions meeting our requirements, such as Equivital TnR suite which is selected as an industry standard instrument in this study \cite{equivital}. Although some work has been undertaken to validate Equivital TnR suite, it is still worthy to compare its performance against another gold standard instrument while being used in various scenarios In this case, Squark b2 from COSMED is selected \cite{cosmed}. Thus, the next objective is 
\begin{itemize}
\item to compare the difference in accuracy between Equivital TnR suite and Cosmed Squark b2 in estimating respiratory frequency while standing and running. 
\end{itemize}