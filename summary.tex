\chapter{Summary and Plan of Future Research}
\section{Summary}
So far, after a large number of relevant research literatures have been evaluated and the analytical methods used in the relevant studies have been considered, this study decided to use the spectrogram generated by the respiratory sound signal to estimate the frequency of breathing with CNN, to determine the stage of breathing and The depth of breathing. The work currently done includes the development of ethics proposal, the collection and labelling of test data, and the use of Colaboratory for prototyping of different signal processing, dataset generation and classifier processes. The results of the tests show that the data sets generated by manual annotation are sufficient for training of classification, and the various methods used in the signal processing flow are very effective. The signal processing flow preserves the respiratory-related frequency components while removing unwanted frequency components. Finally, through the transfer learning, using the Resent model provided by Fastai to distinguish whether a sound is an exhalation sound or an inhalation sound. The performance of the classifier trained with the test data achieved over 50\% accuracy without any fine tuning. After being able to judge the breathing phase, the frequency of breathing can be estimated accordingly.
\section{Plan of future research}
Future research work focuses on preparing dataset, comparing other combinations of signal processing approaches, and various machine learning model, such as support vecter machine and random forest. The details of each of the focuses is discussed in the follow sections.

\subsection{Dataset preparation}

The classifier's performance was likely affected by the inaccurate and uncompleted labelling on the signal. It is crucial to have a clear idea that when the inhalation and exhalation start and end. Thus, the labelling work at next phase will be carried out by aligning the acoustic respiration signal with the reference signal. Also different size of segmentation would be tested.

The purpose of data augmentation is to increase the variety of the dataset especially when the dataset is small. Conventional data augmentation methods includes applying scaling, stratching, cropping on the original images. However, the data augmentation for audio-generated spectrogram image follows different methodologies. Data augmentation has not yet been implemented in this work. However, some research has been done in exploring approaches creating additional data based on existing dataset.\cite{Cho2017DeepBreathSettings} \cite{Schluter2015ExploringNetworks} According to the study undertaken by Schluter and Grill, the most effective augmentation approach is pitch shifting. By also combining the time stretching and random frequency filtering, it could reduced the error signifcantly between 10 and 30\%.\cite{Schluter2015ExploringNetworks}

\subsection{Signal processing}
new signal processing methods like Hilbert huang transform

A results-driven methodology will be applied by evaluating the classification results from different combination of signal processing 

To find the most effective 

\subsection{Feature extraction}
Apart from classifying the respiratory sound based on audio-generated spectrograms it is also one of the objectives to achieve the classification with shallow features. Lei suggested that a combination of different categories of feature outperform a individual catergory of feature.\cite{Lei2014Content-basedFeatures} An enhanced features set including LPC, perceptual and cepstral features were created in their experiment. A number of spectral features are implemented in Librosa library, such as spectral bandwith, spectral mfcc and spectral mfcc delta. The information gain of the features would be evaluated in order to create the most relevant feature sets for the detection of the breathing sound.

\subsection{Classification}
The classifier experimented has show a great potential in discriminating respiratory sound. It is planed to continue to use Resnet34 when formal data is collected at the next phase. More work will be carried on experimenting different hyper-parameters such as learning rate, number of layers and so on to solve the issue of under-fitting at present.

As mentioned above, this study also aims to experiment with other machine learning models for performance comparison, such random forest and support vector machine. 

\subsection{Estimation}
Respiratory frequency is defined as 60/(time interval between current and previous breath (seconds) (5.1))\cite{Mason2002SignalMonitoring}

\subsection{Mobile application development}
Once the prototyped classifier's performance is satisfying, it is aimed to transfer the complete solution to mobile platform to provide respiration-related information in real-time. By taking advantage of FastAI, the model created and trained could be directly exported to Ios system via CoreML with no training required on the mobile devices.

Due to the limitation of resource in computing, computational complexity should be taken into account. Therefore, it requires some optimisation for balance the performance regarding the time taken in classification and its accuracy. For example, down-sampling the signal and downsizing the segmentation are both able to reduce the space required to represent the data. Moreover, the experiment conducted by Niu et al showed a great performance of using grey-scaled spectrogram in classifying breathing signal from subjects with different mental states.\cite{Niu2019AState} So it is valuable to transfer the coloured spectrogram to grey-scaled in future training. 

Excepting manipulating the data size needed to be processed, having a deep learning model with less layers is able to reduce the computational complexity. The research undertaken by Ravi et al suggested a high performance and mobile friendly solution by utilising the combined feature set of features learnt from CNN and shallow features.\cite{Ravi2017ADevices}sThis approach would be take into account if the classifier's performance drops after adjust its structure.



